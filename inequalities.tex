\documentclass[12pt]{article}

\usepackage{answers}
\usepackage{setspace}


\usepackage{multicol}
\usepackage{mathrsfs}
\usepackage[margin=1in]{geometry} 
\usepackage{amsmath,amsthm,amssymb}
 
\usepackage{graphicx}
\usepackage{amsmath,amssymb} % define this before the line numbering.

\usepackage{times}
\usepackage{epsfig}
\usepackage{graphicx}
\usepackage{amsmath}
\usepackage{amssymb}
\usepackage{wrapfig}
\usepackage[export]{adjustbox}
% \usepackage{caption}
\usepackage{color}
\usepackage{enumitem}
\usepackage{hyperref}

\DeclareMathOperator*{\argmax}{arg\,max}
\DeclareMathOperator*{\argmin}{arg\,min}
\usepackage{mathrsfs}
\newcommand*{\QEDA}{\hfill\ensuremath{\blacksquare}}%

% \newcommand{\TODO}[1]{{\color{red} TODO: #1}}

\newcommand\TODO[1]{\fbox{\textcolor{red}{TODO: #1}}}

\newcommand{\specialcell}[2][c]{%
  \begin{tabular}[#1]{@{}c@{}}#2\end{tabular}}

% algorithm env
\usepackage[]{algorithm2e}

% eps support
\usepackage{epstopdf}
\DeclareGraphicsExtensions{.eps}

% subfigure support
\usepackage{subcaption}

% nice tables
\usepackage{booktabs}

\usepackage{algpseudocode}

\newtheorem{theorem}{Theorem}[section]
\newtheorem{prop}{Proposition}[section]
\newtheorem{observation}{Observation}[section]
\newtheorem{corollary}{Corollary}[theorem]
\newtheorem{lemma}[theorem]{Lemma}
\newtheorem{definition}{Definition}[section]

\newcommand{\propautorefname}{Proposition}
\newcommand{\observationautorefname}{Observation}

 
\begin{document}
 
% --------------------------------------------------------------
%                         Start here
% --------------------------------------------------------------
 
\title{MWS and path inequalities}%replace with the appropriate homework number
% \author{Steffen Wolf\\
% Alberto Bailoni}
 
\maketitle
% \subsection{Some definitions}
% Proof by induction that the current defined algorithm is optimal:
% \begin{itemize}
% \item $t=1$: this case is trivial because there are no contraints and the MWS algorithm always add the first edge (here $A_{0} = \emptyset$). Thus $A_{1} = \left\{ e_{1} \right\} \subseteq A^{*}$
% \item Let us assume that $A_{t-1}\subseteq A^*$ and we want to prove $A_{t}\subseteq A^*$
% \end{itemize}
\section{Intro}
\subsection{Some definitions}
\begin{definition}
The vertex labeling $\mathcal{L}$ of $\mathcal{G}$ is defined as a map $\mathcal{L}: V \rightarrow L $, where $L$ is an arbitrary set of labels.
\end{definition}

\begin{definition}
The \emph{boundary} $\partial \mathcal{L}$ of a vertex labeling $\mathcal{L}$ is defined as the edge set 
$$
\partial \mathcal{L} = \left\{ e_{uv}\in E | \mathcal{L}(u) \neq \mathcal{L}(v)\right\}
$$
\end{definition}

\begin{theorem}
For any graph $\mathcal{G}=(V,E)$ and a set of edges $S\subseteq E$, $S$ is a graph cut on $\mathcal{G}$ if and only if there exists a labeling $\mathcal{L}$ s.t. $S=\partial \mathcal{L}$. 
\end{theorem}

\begin{definition}
Let $S$ be a cut on $G$, let $e\in S$,  and let $G^{'}=(V,E\setminus(S\setminus {e}))$. The segment $S^{(e)}$ of $S$ corresponding to $e$ is defined as:
\begin{equation}
S^{(e)} = \left\{ e_{u,v} \big| e_{u,v} \in S, u \underset{G'}{\sim}v   \right\}.
\end{equation}
\end{definition}
\TODO{Missing definition of energy $T$, MWS objective, algorithm, etc...}


\subsection{Properties of the MWS active set}

Some properties of the final active set $A$ defined by the MWS algorithm:
\begin{enumerate}
\item The set $A^{+}$ is a spanning forest on $\mathcal{G}$ and defines a graph cut $S$ and a vertex labeling $\mathcal{L}$.
\item Every edge of the active set \textbf{not on} the final cut $S$ is attractive:
\begin{equation}
\forall e_{uv}\in A \quad \mathrm{s.th.}\quad \mathcal{L}(u)=\mathcal{L}(v) \quad \Longrightarrow \quad e_{uv} \in E^+
\end{equation}
\item Every edge of the active set \textbf{on} the final cut $S$ is repulsive:
\begin{equation}
\forall e_{uv} \in A \quad \mathrm{s.th.}\quad \mathcal{L}(u)\neq\mathcal{L}(v) \quad \Longrightarrow \quad e_{uv} \in E^-
\end{equation}



\item All repulsive edges on the final cut belong to the active set: ${A^{-} = S \cap E^{-}}$. %(it follows from the algorithm definition \TODO{or the energy..?}).
\item From the properties of the spanning forest $A^+$, it follows that for every pair of vertices $u,v\in V$ in the same connected component, it exists only one path connecting them that is included in $A^+$:
\begin{equation}
\forall u,v \in V\quad \mathrm{s.th.} \, \mathcal{L}(u)=\mathcal{L}(v)\quad  \Longrightarrow \quad\exists ! \, \mathrm{path} \, \pi_{u \rightarrow v}\subseteq A^+
\end{equation}

\item Let us assume to have a cycle $c$ from the set $C_{0/1}(A)$, given some set $A$. Then if we consider the vertex labeling $\mathcal{L}$ associated to the spanning forest $A\cap E^+$, we have that every vertex connected by the cycle is in the same connected component:

\begin{equation}
\forall e_{uv} \in c \quad \Rightarrow \quad \mathcal{L}(u) = \mathcal{L}(v) 
\end{equation}

\item Even better property: Let us consider an active set $A$ such that $C_{0/1}(A) =  \emptyset$ and two vertices $u,v$ not connected by the set $A$: such that $u \underset{(V,A)}{\not\sim}v$. Then we can always add an edge $e = (u,v)$ to the set $A$ without violating any constraint: $C_{0/1}(A \cup \{e\}) =  \emptyset$.
\end{enumerate}


\begin{prop} \label{prop:A_connected}
If we assume that the graph $\mathcal{G}$ is connected, then the final active set $A$ defines a connected subgraph of $\mathcal{G}$, i.e. $u \underset{(V,A)}{\sim}v$ for every pair of vertices $u,v \in V$. 
\end{prop}
\begin{proof}
%Since $\mathcal{G}$ is connected, we note that $\forall u,v\in V \,\mathrm{s.t.}\, \mathcal{L}(u)=\mathcal{L}(v)$, then $u \underset{\mathcal{G}}{\sim}v$.
Let us introduce the cut $S$ defined by the connected components of the spanning forest  $A^{*} \cap E^+$. To prove the theorem, we need to show that every boundary of the cut $S$ includes at least one edge of $A^*$:
\begin{equation}
\forall \tilde{e} \in S \quad \Longrightarrow \quad S^{(\tilde{e})}\cap A^* \neq \emptyset.
\end{equation}

To prove this proposition, it is enough to show that every segment of the cut $S$ includes at least one edge in the active set $A^- \subseteq E^-$. 
%(every connected component defined by the spanning forest $A^+$ is connected by some repulsive edges). 
By contradiction, let us assume that:
\begin{equation}
\exists \, \tilde{e} \in S \quad \mathrm{s.th.} \quad S^{(\tilde{e})}\cap E^- = \emptyset.
\end{equation}
Thus, the edge $\tilde{e}$ has to be attractive and we can construct an new feasible active set $A'= A \cup \{\tilde{e}\}$ not violating any path-constraints. \TODO{Explain better} We then note that $T(A')>T(A)$, which is a contradiction. 

%and we could add it to the active set $A^+$ without violating any loop-constraints. In this way we would find a new feasible solution $A \cup \{\tilde{e}\}$ with an higher energy than the optimal solution $A$, which is a contradiction. 
\end{proof}


\section{Reduced path inequalities}
\begin{theorem}
Let us consider the optimal active set $A^*$ found by the MWS algorithm on a graph $\mathcal{G} = (V, E^+ \cup E^-)$. Then, for every edge $\tilde{e}_{uv} \in E\setminus A^*$ not in the optimal active set, 
it exists a path $\pi_{u \rightarrow v}  \subseteq A^* $ from $u$ to $v$ such that the weight of its weakest edge is higher than the weight of $\tilde{e}_{uv}$:
% \begin{enumerate}
% \item the path is : $\pi_{u \rightarrow v}$,  \label{cond_1} 
% % \item the path includes exactly zero or one repulsive edges: $\left| \pi_{u \rightarrow v} \cap E^{-}  \right| \leq 1$, \label{cond_2} 
% \item the weight of the weakest edge along the path is higher than the weights of $\tilde{e}$: \label{cond_3}
\begin{equation}
\forall \tilde{e}_{uv} \in E\setminus A^*, \quad \exists \,\mathrm{path}\, \pi_{u \rightarrow v}  \subseteq A^* \quad \mathrm{s.th.} \quad w(\tilde{e}_{uv}) < \min_{p\in \pi_{u \rightarrow v}}  w(p). \label{eq:reduced_path_ineq}
\end{equation} 

% \end{enumerate} 
\end{theorem}
\begin{proof}
Since the active set $A^*$ is optimal, it follows that the new active set $A' = A^* \cup \{ \tilde{e}_{uv}\}$ has to violate some cycle constraints:
\begin{equation}
C_{0/1}(A^* \cup \{ \tilde{e}_{uv}\}) \neq \emptyset.
\end{equation}
This shows that %$(V, A^*)$ is a connected subgraph of $\mathcal{G}$ and 
it always exists one path in the active set $A^*$ connecting the vertices $u$ and $v$.

We will now prove that there is at least one of these paths fulfilling the property in Eq. \ref{eq:reduced_path_ineq}. To do so, let us define the set of weakest edges along each of the violating cycles in $C_{0/1}(A')$:

\begin{equation}
\mathcal{W} \equiv \left\{ e \in E \quad \Big| \quad\exists \, c\in C_{0/1}(A^* \cup \{ \tilde{e}_{uv}\}) \quad \mathrm{s.th.} \quad e = \argmin_{p\in c} w(p) \right\}.
\end{equation}
In order to prove the theorem, we need to show that $\tilde{e}_{uv} \in \mathcal{W}$, i.e. $\tilde{e}_{uv}$ is the weakest edge of at least one violating cycle. In this case, this violating cycle defines one path $\pi_{u \rightarrow v}$ in the active set $A^*$ connecting $u$ and $v$ fulfilling property \ref{eq:reduced_path_ineq}:
\begin{equation}
w(\tilde{e}_{uv}) < \min_{p\in \pi_{u \rightarrow v}}  w(p).
\end{equation}

We prove now by contradiction that $\tilde{e}_{uv} \in \mathcal{W}$. If we assume the opposite, i.e. $\tilde{e}_{uv} \notin \mathcal{W}$, it follows that:
\begin{equation}
w(\tilde{e}_{uv}) > w(e), \quad \quad \forall \,e\in \mathcal{W} \label{eq:ineq_weakest_edges}
\end{equation}
and from the properties of the energy function $T$ it follows that:
\begin{equation}
T(\{\tilde{e}_{uv}\}) = 2^{w(\tilde{e}_{uv})} \overset{(\ref{eq:ineq_weakest_edges})}{>} \sum_{e \in \mathcal{W}} 2^{w(e)} = T( \mathcal{W} ) \label{eq:strict_energy_ineq}
\end{equation}
If we now construct a new active set $A^{''}=A^*\cup\{\tilde{e}_{uv}\} \setminus \mathcal{W}$, we note that it does not violate any cycle constraint and it has a higher energy than the optimal active set $A^*$:
\begin{equation}
T(A'') = T(A^*) + T(\{\tilde{e}_{uv}\}) - T( \mathcal{W} ) \overset{(\ref{eq:strict_energy_ineq})}{>}T(A^*).
\end{equation}
We have then a contradiction.
% We prove the theorem by showing that it holds for every edge $\tilde{e}_{uv}$ not belonging to the active set $A$. \\
% \noindent \textbf{Case 1: $\tilde{e}_{uv}$ is an edge not on the final cut $S$} 
% In this case $\tilde{e}_{uv} \in S$ and $\mathcal{L}(u) = \mathcal{L}(v)$ (see Fig. \ref{fig:proof_case_1}).
% Property (4...?) of the spanning forest $A^+$ proves the existence of a unique positive path ${\pi_{u \rightarrow v}\subseteq A^+}$. \\
% We now prove by contradiction that $\pi_{u \rightarrow v}$ fulfills condition \hyperref[cond_3]{3}. Let us assume that $\exists \, p\in \pi_{u \rightarrow v}$ such that $w(p)< w(\tilde{e}_{uv})$ and we then construct an new feasible active set $A'= A \cup \{\tilde{e}_{uv}\} \setminus \{p\}$ not violating any path-constraints. We then note that $T(A')>T(A)$, which is a contradiction. 
% \TODO{Explain better this last part}\\
% \noindent \textbf{Case 2: $\tilde{e}_{uv}$ is an edge on the final cut $S$}  
% In this case $\tilde{e}_{uv} \in S $ and $\mathcal{L}(u)\neq\mathcal{L}(v)$ (see Fig. \ref{fig:proof_case_2}).
% From Property (3...?) of the active set, we know that all repulsive edges on the final cut belong to the active set. Thus, in this case $\tilde{e}_{uv}$ has to be an attractive edge.\\ 
% Let us define the set $\Pi_{u \rightarrow v}^-$ of all paths connecting $u$ and $v$ that are included in $A$ and includes exactly one repulsive edge: 
% \begin{equation}\label{eq:set_of_paths_in_A}
% \Pi_{u \rightarrow v}^- \equiv \left\{ \mathrm{path} \,\, \pi_{u \rightarrow v}  : \pi_{u \rightarrow v} \subseteq A, \, \left| \pi_{u \rightarrow v} \cap E^{-}  \right| \leq 1 \right \}.
% \end{equation}  
% The edge $\tilde{e}_{uv}$ connects two neighboring connected components of the spanning forest $A^+$. Since $\mathcal{G}$ is connected, from \autoref{prop:A_connected} it follows that $\Pi_{u \rightarrow v}^- \neq \emptyset$. Let us now define the weakest edge $\bar{p}(\pi)$ of a path $\pi$:
% \begin{equation}
% \bar{p}(\pi) \equiv \argmin_{p\in \pi}  (w(p))
% \end{equation}
% and the set of the weakest edges of all paths defined in (\ref{eq:set_of_paths_in_A}) connecting $u$ and $v$:
% \begin{equation}
% \bar{P}_{u \rightarrow v} = \left\{ \bar{p}(\pi) \, \big| \,\pi \in \Pi_{u \rightarrow v}^- \right\}.
% \end{equation}
% We now want to prove that it exists at least one path in $\Pi_{u \rightarrow v}^-$ fulfilling condition \hyperref[cond_3]{3}. By contradiction let us assume the opposite and that edge $\tilde{e}_{uv}$ is stronger than all other paths in $\Pi_{u \rightarrow v}^-$ connecting $u$ and $v$:
% \begin{equation}
% w(\tilde{e}_{uv}) > w(p), \quad \quad \forall p \in \bar{P}_{u \rightarrow v}.
% \end{equation}
% In this case we can then construct an new feasible active set $A'= A \cup \{\tilde{e}_{uv}\} \setminus \bar{P}_{u \rightarrow v}$ not violating any path-constraints. We then note that $T(A')>T(A)$, which is a contradiction.  
\end{proof}


\end{document}
