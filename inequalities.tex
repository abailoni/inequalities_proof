\documentclass[12pt]{article}

\usepackage{answers}
\usepackage{setspace}


\usepackage{multicol}
\usepackage{mathrsfs}
\usepackage[margin=1in]{geometry} 
\usepackage{amsmath,amsthm,amssymb}
 
\usepackage{graphicx}
\usepackage{amsmath,amssymb} % define this before the line numbering.

\usepackage{times}
\usepackage{epsfig}
\usepackage{graphicx}
\usepackage{amsmath}
\usepackage{amssymb}
\usepackage{wrapfig}
\usepackage[export]{adjustbox}
% \usepackage{caption}
\usepackage{color}
\usepackage{enumitem}
\usepackage{hyperref}

\DeclareMathOperator*{\argmax}{arg\,max}
\DeclareMathOperator*{\argmin}{arg\,min}
\usepackage{mathrsfs}
\newcommand*{\QEDA}{\hfill\ensuremath{\blacksquare}}%

% \newcommand{\TODO}[1]{{\color{red} TODO: #1}}

\newcommand\TODO[1]{\fbox{\textcolor{red}{TODO: #1}}}

\newcommand{\specialcell}[2][c]{%
  \begin{tabular}[#1]{@{}c@{}}#2\end{tabular}}

% algorithm env
\usepackage[]{algorithm2e}

% eps support
\usepackage{epstopdf}
\DeclareGraphicsExtensions{.eps}

% subfigure support
\usepackage{subcaption}

% nice tables
\usepackage{booktabs}

\usepackage{algpseudocode}

\newtheorem{theorem}{Theorem}[section]
\newtheorem{corollary}{Corollary}[theorem]
\newtheorem{lemma}[theorem]{Lemma}
\newtheorem{definition}{Definition}[section]

 
\begin{document}
 
% --------------------------------------------------------------
%                         Start here
% --------------------------------------------------------------
 
\title{MWS and path inequalities}%replace with the appropriate homework number
% \author{Steffen Wolf\\
% Alberto Bailoni}
 
\maketitle
% \subsection{Some definitions}
% Proof by induction that the current defined algorithm is optimal:
% \begin{itemize}
% \item $t=1$: this case is trivial because there are no contraints and the MWS algorithm always add the first edge (here $A_{0} = \emptyset$). Thus $A_{1} = \left\{ e_{1} \right\} \subseteq A^{*}$
% \item Let us assume that $A_{t-1}\subseteq A^*$ and we want to prove $A_{t}\subseteq A^*$
% \end{itemize}
\section{Some definitions}

\begin{definition}
The vertex labeling $\mathcal{L}$ of $\mathcal{G}$ is defined as a map $\mathcal{L}: V \rightarrow L $, where $L$ is an arbitrary set of labels;
\end{definition}

\begin{definition}
The \emph{boundary} $\partial \mathcal{L}$ of a vertex labeling $\mathcal{L}$ is defined as the edge set $\partial \mathcal{L} = \left\{ e_{u,v}\in E | \mathcal{L}(u) \neq \mathcal{L}(v)\right\}$
\end{definition}

\begin{theorem}
It is easy to show that for any graph $\mathcal{G}=(V,E)$ and a set of edges $S\subseteq E$, $S$ is a graph cut on $\mathcal{G}$ if and only if there exists a labeling $\mathcal{L}$ s.t. $S=\partial \mathcal{L}$. 
\end{theorem}

\begin{definition}
Let $S$ be a cut on $G$, let $e\in S$,  and let $G^{'}=(V,E\setminus(S\setminus {e}))$. The segment $S^{(e)}$ of $S$ corresponding to $e$ is defined as:
\begin{equation}
S^{(e)} = \left\{ e_{u,v} \big| e_{u,v} \in S, u \underset{G'}{\sim}v   \right\}.
\end{equation}
\end{definition}


\section{Path inequalities}
\begin{theorem}
(Assume for simplicity and without loss of generality that $\mathcal{G}$ is connected). If we run the MWS algorithm until termination, we get a final active set ${A = A^{+} \cup A^{-}}$, where $A^{+} \subseteq E^+$ and $A^{-} \subseteq E^-$. For every edge $\tilde{e}=(u,v)$ not in the active set, i.e. $\tilde{e} \in E\setminus A^*$, it exists a path $\pi$ from $u$ to $v$ such that:
\begin{enumerate}
\item the path is included in the active set: $\pi \subseteq A^*$,  \label{cond_1} 
\item the path includes exactly zero or one repulsive edges: $\left| \pi \cap E^{-}  \right| \leq 1$, \label{cond_2}
\item the weight of the weakest edge along the path is higher than the weights of $\tilde{e}$: \label{cond_3}
\[ 
w(\tilde{e}) < \min_{p\in \pi}  (w(p)).
\]
\end{enumerate} 
\end{theorem}
\begin{proof}



Let us define:
\begin{itemize}
\item The set $A^{+}$ is a spanning forest on $\mathcal{G}$ and defines a graph cut $S$ and a vertex labeling $\mathcal{L}$.
\end{itemize}

Some properties of the active set and the final cut:
\begin{itemize}
    \item We note that the repulsive active set $A^{-}$ is simply given by all the repulsive edges on the final cut: $A^{-} = S \cap E^{-}$ (it follows from the algorithm definition).
\end{itemize}

\begin{theorem}
Prova
\end{theorem}
\begin{proof}
Ciao
\end{proof}


\TODO{I could also use $\underset{(V,A^{*})}{u\thicksim v}$}

Depending on the type of edge $\tilde{e}$, we consider three different cases and we show that the theorem holds for all of them.

\noindent \textbf{Case 1: Attractive edge not in the active set and NOT on final cut} 

$\tilde{e} \in E^+ \setminus S^* $

 $A^* \cap E^+$ defines a spanning forest on $(V,E)$ associated to the cut $S^*$. Thus, the existence of a path $\pi \subseteq A^* \cap E^+$ from $u$ to $v$ directly follows from $\tilde{e} \notin S^* $ (see Fig. \ref{fig:proof_case_1}). Particularly, there is only one path fulfilling conditions \hyperref[cond_1]{1} and \hyperref[cond_2]{2}.  

\noindent \textbf{Case 2: Attractive edge not in the active set and on final cut}  

$\tilde{e} \in E^+ \cap S^* $


\noindent \textbf{Case 3: Repulsive edge not in the active set} 

$\tilde{e} \in E^-$


First, we note that the only repulsive edges in the active set $A^*$ are those on the cut $S^*$:
\begin{equation}
E^- \cap S^* = E^- \cap A^*.
\end{equation}
 Thus, in this third case $\tilde{e} \notin S^*$ and all paths $\pi$ \TODO{introduce a new notation..?} fulfilling conditions \hyperref[cond_1]{1} and \hyperref[cond_2]{2} do not include any repulsive edge (see Fig. \ref{fig:proof_case_3}). 
\end{proof}


\end{document}
